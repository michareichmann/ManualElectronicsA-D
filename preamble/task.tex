% DEFINE LIST
\newcommand{\listtaskname}{List of Tasks} % Set the name of the list

\newlistof{tasks}{tsk}{\listtaskname} % First bracket is \listof{what's in here}, second is the document that will be created, NameOfYourDoc.tsk in this case. Third is title of the list, what you define in above \newcommand.
%
\newcommand{\tasks}[1]{% How to set a counter for. 
\refstepcounter{tasks} %increase the counter
\par\noindent\textbf{Task \thetasks:~#1} %The text that will be used when \tasks is used.
\addcontentsline{tsk}{tasks} % Add the line to the .tsk file
{\protect\numberline{\thetasks}#1}\par} %I do not understand this, maybe somebody can explain.

\setlength{\cfttasksindent}{1.5em} %Indent of the example #. Same as \listoffigures
\setlength{\cfttasksnumwidth}{2.3em} %Indent of the tasks title. Same as \listoffigures
\renewcommand{\cfttsktitlefont}{\bfseries\Large} % Sets the font for \listtaskname

%
% DEFINE ENVIRONMENT
\mdfsetup{skipabove=\topskip, skipbelow=\topskip}
\newenvironment{task}[1][]{%
\ifstrempty{#1}
{\mdfsetup{frametitle=\null\hspace*{1em}{%
\tikz[baseline=(current bounding box.east),outer sep=0pt, rounded corners=2mm]
\node[anchor=east,rectangle,fill=red!30, draw=red, line width=1pt]
{\tasks{#1}};}}%
}
{\mdfsetup{frametitle=\null\hspace*{1em}{%
\tikz[baseline=(current bounding box.east),outer sep=0pt, rounded corners=2mm]
\node[anchor=east,rectangle,fill=red!30, draw=red, line width=1pt]
{\tasks{#1}};}}%
}
%
\mdfsetup{roundcorner=5pt, frametitleaboveskip=-12pt, frametitlebelowskip=1pt, innerleftmargin=1em, linecolor=red!50, backgroundcolor=red!5}

\begin{mdframed}[]\relax\bfseries%
}{\end{mdframed}}
