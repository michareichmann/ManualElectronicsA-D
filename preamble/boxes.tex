\mdfsetup{skipabove=\topskip, skipbelow=\topskip}
%
\newcommand{\setboxframe}[1]{
  \mdfsetup{roundcorner=5pt, frametitleaboveskip=-12pt, frametitlebelowskip=1pt, innerleftmargin=1em, linecolor=#1!50, backgroundcolor=#1!5}
}
%
\newcommand{\makeboxtitle}[2]{
  \mdfsetup{frametitle=\null\hspace*{1em}{%
    \tikz[baseline=(current bounding box.east),outer sep=0pt, rounded corners=2mm]
    \node[anchor=east, rectangle, fill=#2!30, draw=#2, line width=1pt]
    {#1};}%
  }%
}
%
% NORMAL BEGIN ----------------------------------------
% DEFINE LIST
\newcommand{\listtaskname}{List of Tasks} % Set the name of the list
\newlistof{tasks}{tsk}{\listtaskname} % First bracket is \listof{what's in here}, second is the document that will be created, NameOfYourDoc.tsk in this case. Third is title of the list, what you define in above \newcommand.
%
\newcommand{\tasks}[1]{% How to set a counter for. 
\refstepcounter{tasks} %increase the counter
\par\noindent\textbf{Task \thetasks\ifstrempty{#1}{}{:~#1}} %The text that will be used when \tasks is used.
\addcontentsline{tsk}{tasks} % Add the line to the .tsk file
{\protect\numberline{\thetasks}#1}\par %I do not understand this, maybe somebody can explain.
} 
\setlength{\cfttasksindent}{1.5em} %Indent of the example #. Same as \listoffigures
\setlength{\cfttasksnumwidth}{2.3em} %Indent of the tasks title. Same as \listoffigures
\renewcommand{\cfttsktitlefont}{\bfseries\Large} % Sets the font for \listtaskname
%
% DEFINE ENVIRONMENT
\newenvironment{task}[1][]{%
  \makeboxtitle{\tasks{#1}}{red}
  \setboxframe{red}
%
  \begin{mdframed}[]\relax\bfseries%
}{\end{mdframed}}
% NORMAL END ------------------------------------------
%
% ADVANCED BEGIN --------------------------------------
\newcommand{\listataskname}{List of Advanced Tasks} % Set the name of the list
\newlistof{atasks}{atk}{\listataskname}
%
\newcommand{\atasks}[1]{
  \refstepcounter{atasks}%
  \par\noindent\textbf{Advanced Task \theatasks:~#1}%
  \addcontentsline{atk}{atasks}{\protect\numberline{\theatasks}#1}\par%
} 
\setlength{\cftatasksindent}{1.5em} 
\setlength{\cftatasksnumwidth}{2.3em} 
\renewcommand{\cftatktitlefont}{\bfseries\Large} 

\newenvironment{atask}[1][]{%
  \makeboxtitle{\atasks{#1}}{MidnightBlue}
  \setboxframe{MidnightBlue}
%
  \begin{mdframed}[]\relax\bfseries%
}{\end{mdframed}}
% ADVANCED END ----------------------------------------
%
% NOTES BEGIN -----------------------------------------
\newcommand{\listnotename}{List of Notes} % Set the name of the list
\newlistof{notes}{nte}{\listnotename}
% 
\newcommand{\notes}[1]{
  \refstepcounter{notes}%
  \par\noindent\textbf{Note \thenotes:~#1}%
  \addcontentsline{nte}{notes}{\protect\numberline{\thenotes}#1}\par%
} 
\setlength{\cftnotesindent}{1.5em} 
\setlength{\cftnotesnumwidth}{2.3em} 
\renewcommand{\cftntetitlefont}{\bfseries\Large} 

\newenvironment{note}[1][]{%
  \makeboxtitle{\notes{#1}}{OliveGreen}
  \setboxframe{OliveGreen}
%
  \begin{mdframed}[]\relax\bfseries%
}{\end{mdframed}}
% NOTES END -------------------------------------------

