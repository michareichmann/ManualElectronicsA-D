\section{Setup and Experimental Procedure}\label{sec:exp}
The Digital Electronics Lab is meant to be very open in implementation. The aim will be to build an automated cooling system. We will guide you through this process step by step, but you are also encouraged to bring in your own ideas of other possible implementations!

\subsection{Experimental Material}\label{sec:material}
%BEGIN Material
You will find the following objects on your test stand:
\begin{itemize}
	\item 1 Arduino Uno Board
	\item 1 Oscilloscope
	\item 2 Probes 
	\item 1 Multimeter
	\item 1 \ac{USB} Type B cable
	\item 1 Breadboard
	\item 1 Grove Starter Kit for Arduino
	\begin{itemize}
		\item 1 Base Shield
		\item 1 LCD RGB Backlight
		\item 1 Smart Relay
		\item 1 Buzzer
		\item 1 Sound Sensor
		\item 1 Touch Sensor
		\item 1 Rotary Angle Sensor
		\item 1 Temperature Sensor
		\item 1 LED Module
		\item 3 Dip LEDs (red, green, blue)
		\item 1 Light Sensor
		\item 1 Button
		\item 1 Mini Servo
		\item 10 Grove Cables
	\end{itemize}
	\item 1 3-pin fan
	\item 1 operational amplifier (\href{http://www.ti.com/lit/ds/symlink/lm158-n.pdf}{LM358})
	\item 1 NPN transistor (\href{https://www.sparkfun.com/datasheets/Components/BC546.pdf}{BC547})
	\item 1 \ac{NTC} thermistor (\href{https://eu.mouser.com/datasheet/2/400/NTC_Leaded_disks_K164-1317145.pdf}{B57164-K104-J})
\end{itemize}\vspace*{10pt}\noindent
\textcolor{red}{\textbf{In addition you should bring a lab book to note down everything what you do and measure (immediately after the execution). Please write clearly and meticulously since it will greatly help you to find mistakes and to write your report!}}
%END

\subsection{Setting up the Arduino}\label{sec:setup}
%BEGIN Arduino Setup
\subsubsection{Software installation}
We highly recommend you to install the Arduino \ac{IDE} on your machine so that you can access it any time. If you should have trouble with the installation or would prefer to work you can also use the online \ac{IDE}.
\begin{itemize}
	\item install \ac{IDE} following \ar{sec:soft} (Do we provide some laptops just in case?)
\end{itemize}

\subsubsection{Connect the Arduino}
\begin{itemize}
	\item connect the Arduino to your computer and check if it is recognised by the \ac{IDE}
	\begin{itemize}
		\item select the correct port: \fpath{Tools > Port}
		\item get board info: \fpath{Tools > Get Board Info}
	\end{itemize}
\end{itemize}

\subsubsection{Your First Sketch}
\begin{itemize}
	\item create your sketch directory
	\begin{itemize}
		\item you can link it to your \ac{IDE}: \fpath{File > Preferences > Sketchbook Location}
	\end{itemize}
	\item create a git repository for this project in the sketch folder
	\item write a sketch that turns on the built in LED 
	\begin{itemize}
		\item name it \path{LedOn.ino}
		\item \code{\var{const int} LedPin = \var{LED\_BUILTIN};}
	\end{itemize}
\end{itemize}
%END

\subsection{Blinking LED on Bread Board}\label{sec:led}
%BEGIN Blinking LED
This task will teach you how to use the Breadboard with the Arduino and the difference between \code{\meth{delay}()} and \code{\meth{millis}()}.
\begin{itemize}
	\item supply the Breadboard with \SI{5}{\volt} from the Arduino
	\item connect a LED to the Breadboard 
	\item write a sketch (\path{Blink.ino}) that makes the LED blink with a frequency of \SI{10}{\hertz} using the \code{\meth{delay}()} method
	\item connect a second LED to the circuit
	\item write a sketch (\path{Blink2.ino}) that makes the second LED blink in a different pattern than the first one
	\begin{itemize}
		\item why is \code{\meth{delay}()} not working anymore?
		\item use a timer with \code{\meth{millis}()} or \code{\meth{micros}()}
	\end{itemize}
\end{itemize}
%END


\subsection{Grove Temperature Sensor}\label{sec:grovetemp}
%BEGIN Grove Temp
In this experiment you will work with the \href{http://wiki.seeedstudio.com/Grove-Temperature_Sensor_V1.2/}{Grove - Temperature Sensor V1.2}. It uses a \ac{NTC} thermistor to detect the ambient temperature. The specifications are shown in \ar{tab:gt}.
\begin{table}[ht!]\centering\alternatecolors
	\begin{tabular}{|ll|}\rowcolor{PineGreen}\tline{.5}
		\fatwhite{Specification}		& \fatwhite{Value}																					\\\tline{1.3}
		Operating voltage						&	\SIrange{3.3}{5.0}{\volt}																	\\
		Zero power resistance				&	\SI{100\pm1}{\kilo\ohm}																		\\
		Operating temperature range	&	\SIrange[retain-explicit-plus]{-40}{+125}{\degreeCelsius}	\\
		Nominal $B$-constant				&	\SIrange{4250}{4299}{K}																		\\\tline{.4}
	\end{tabular}
	\caption{Grove-Temperature Sensor V1.2 specifications.}
	\label{tab:gt}
\end{table}

\begin{itemize}
	\item connect the Grove Base Shield to your Arduino
	\item connect the Grove Temperature Sensor to the shield
	\begin{itemize}
		\item think about which connector to choose on the shield
		\item \code{\var{const int} TempPin = D8/A2;}?
	\end{itemize}
	\item write a sketch (\path{GroveTemp.ino}) that measures the ambient temperature
	\begin{itemize}
		\item read the voltage value from the sensor
		\item convert the voltage into the corresponding resistance
		\item convert the resistance into a temperature in \SI{}{\degreeCelsius}
		\item write the temperature to the serial interface
		\item look at the data with the Arduino Serial Monitor:\\ \fpath{Tools > Serial Monitor} (\code{Ctrl+Shift+M})
		\item look at the data with the Arduino Serial Plotter:\\ \fpath{Tools > Serial Plotter} (\code{Ctrl+Shift+P})
	\end{itemize}
\end{itemize}
%END


\subsection{Grove Display and Potentiometer}\label{sec:grovetemp}
%BEGIN Grove3
As a next step, we will add two more components from the Grove kit: a display to output the current temperature read by the sensor and a potentiometer to set the threshold for out two-point temperature control later. 
\begin{itemize}
    \item add the display to your project
	\begin{itemize}
        \item install the libraries from \href{http://wiki.seeedstudio.com/Grove-LCD_RGB_Backlight/}
        \item look at the example code on the above website, especially on how to include the libraries in your project and how to initialize it during the \code{setup()} routine
        \item make sure the "3V3\_VCC\_5V" switch on the Grove shield is set to 5V, otherwise the display will not work correctly
	    \item connect the Grove LCD RGB Backlight display to your Grove shield
		\item modify your project such that the measured temperature is printed on the display every second 
	\end{itemize}
	\item define a fixed threshold in your code, e.g. 30 degrees above which a LED is turned on. (As alternative you can change the background color of your display). Test if your project is working.
	\item now we want to make the threshold controllable without recompiling our project. For this purpose, connect the potentiometer (Grove Rotary Angle Sensor) to the Grove shield.
	\begin{itemize}
        \item read the description at \href{http://wiki.seeedstudio.com/Grove-Rotary_Angle_Sensor/}
        \item read-out the sensor with \code{analogRead()}
        \item to map the ADCs which range from 0 to 1023 to a reasonable temperature range (a, b), e.g. 20 to 40 degrees, the function \code{map(degrees, 0, 1023, a, b)} can be used for convenience
 	\end{itemize} 
 	\item indicate the status below/above threshold also in the output to the serial interface, e.g. add a second number (0 = below threshold, 50 = above threshold) separated by a space
    	\item test your code by varying the threshold below/above the room temperature
	\begin{itemize}
		\item the serial plotter should now draw a second line indicating whether below or above threshold
 	\end{itemize}
    	\item optional: while turning the potentiometer, you can also print the threshold temperature on the display in the second row below the measured temperature
\end{itemize}
%END

\subsection{Data Handling}\label{sec:data}
%BEGIN Data Handling
Even though one can use the Arduino \ac{IDE} to look at the values from the temperature sensor, there is no way to save them. That is why we encourage you to write a short program to save the data to file so that you can use it afterwards. We recommend you to use python for this purpose, but feel free to use whatever you have the most experience with.\par
Once a sketch is uploaded to the microprocessor the Arduino will perform the loop until it is disconnected from power or overwritten by a new sketch. In order to read the data externally from the serial port you have to close the Arduino serial tools.\\
Example code in python:
\begin{itemize}
	\item open the serial port
	\begin{itemize}
		\item \code{\var{from} serial \var{import} Serial}
		\item \code{arduino = Serial(\str{'/dev/<arduino\_portname>'})}
		\item you can find the name of the port in the Arduino \ac{IDE}
		\item read a single line using the serial method \code{arduino.readline().decode('utf-8')}  
	\end{itemize}
	\item print the line in the terminal using \code{print}
	\item write the data to a text file (\path{data.txt}), every value on a single line
	\begin{itemize}
		\item if you have trouble handling files in python follow this \href{http://www.pythonforbeginners.com/files/reading-and-writing-files-in-python}{guide}
	\end{itemize}
	\item your program should repeat this until you terminate it e.g. by pressing CTRL+C.
\end{itemize}
\b{Optional}: If you use Linux or Mac, try to do all of the above with a single line of bash with less than 100 characters. Look at \code{read} and \code{tee} commands.
%END

\subsection{Building Your Own Temperature Sensor}\label{sec:temp}
%BEGIN Temperature Sensor
In this task you will build your own temperature sensor using discrete components. The \href{https://eu.mouser.com/datasheet/2/400/NTC_Leaded_disks_K164-1317145.pdf}{B57164-K104-J} thermistor has the specifications listed in \ar{tab:ts}.
\begin{table}[ht!]\centering\alternatecolors
	\begin{tabular}{|ll|}\rowcolor{PineGreen}\tline{.5}
		\fatwhite{Specification}		& \fatwhite{Value}																					\\\tline{1.3}
		Operating voltage											&	\SIrange{3.3}{5.0}{\volt}																	\\
		Zero power resistance				&	\SI{100\pm5}{\kilo\ohm}																		\\
		Operating temperature range	&	\SIrange[retain-explicit-plus]{-55}{+125}{\degreeCelsius}	\\
		Nominal $B$-constant				&	\SI{4600\pm1340}{K}																		\\\tline{.4}
	\end{tabular}
	\caption{B57164-K104-J specifications.}
	\label{tab:ts}
\end{table}

\begin{itemize}
	\item build a voltage divider circuit to convert the resistance into a measurable voltage
	\item use the operational amplifier to increase the outgoing signal
	\item reproduce the temperature measurements from \ar{sec:grovetemp}
\end{itemize}
%END


\subsection{Building a Heating System}\label{sec:heat}
Since it is boring to measure a constant temperature you build a system to heat your temperature sensor.
\begin{itemize}
	\item connect a resistor (or a series of resistors) to the Breadboard
	\item calculate the power dissipation of the resistor
	\item heat the temperature sensor with the heater
	\item monitor the temperature
	\item measure time constant of the temperature rise?
\end{itemize}

\subsection{Building a Cooling System}\label{sec:cool}
Many electrical components will produce heat under load and may break at critical temperatures. That is why many of complex systems like computers require cooling. You will now build a system that controls a fan and can regulate it's rotation speed.
\begin{itemize}
	\item connect the fan to the breadboard, for the 4 pin header, the following conventions are used:
	\begin{itemize}
	    \item black: ground
	    \item yellow: 5V
	    \item green: tacho read-out (will be used later)
	    \item blue: PWM control
	\end{itemize}
	\item keep in mind that only a few of the digital pins (marked by "\~" on the board) are able to use PWM, e.g. use digital pin 11
\end{itemize}
Now we can finally build the full two-point controller.
\begin{itemize}
    \item define a low and high threshold, e.g. use the potentiometer for the high threshold and set the lower threshold a few degrees lower than the high one.
	\item set the duty cycle of the fan to 100\% when above the high threshold
	\item set the duty cycle of the fan to 0\% when below the low threshold
	\item setting the PWM duty-cycle can be done using the \code{analogWrite()} function
	\item if thresholds are set reasonably, you will see an oscillatory behavior. \textbf{Create a plot of temperature vs. time which shows multiple periods.}
\end{itemize}


\subsection{Read Out the Fan Speed}
We now wan to use the built-in Hall Effect Sensor (HES) of the fan to measure its rotation speed. In every rotation this sensor will produce two pulses we can count and then convert to revolutions per minute (\textbf{RPM}). The maximum (for 100\% duty cycle) according to the datasheet is 1900 RPM, with a tolerance of 10\%, the minimum is around 230 RPM.

\begin{itemize}
	\item calculate a rough estimate for the minimum frequency needed for reading the voltage on the tacho pin and be able to count the pulses? (Use Nyquist theorem) In practice, a much higher frequency should be used than the limit calculated above
	\item check the Arduino reference of \code{analogRead()} for the maximum frequency. Since we also have to do other work in the \code{loop()} function, the frequency should be also much less than the maximum. Make a reasonable choice.
    \item what is the expected uncertainty on the minimum and maximum RPM if we count pulses for 1 second?
	\item connect the tacho (green wire) of the fan to an analog pin of the Arduino, using a 10k pull-up resistor. (optional: figure out how to use internal pull-up resistor of the Arduino instead of a discrete component)
	\item what is the role of the pull-up resistor?
	\item count the pulses of the HES and convert the result to RPM. Is it within the range expected by the datasheet numbers?
	\item write the \ac{RPM} value to the serial output
\end{itemize}

\subsection{Final measurements}
Now that our device has all basic functionalities, we can perform some measurements with it.
\begin{itemize}
	\item plot the fan RPM vs. the duty cycle. Which is the minimum duty cycle above which the fan starts to spin?
	\item plot the equilibrium temperature vs. the duty cycle and the equilibrium temperature vs. the fan RPM
\end{itemize}

Scanning the points for the different duty cycles should be done without flashing the device in between. Instead, the measurement programme (e.g. number of steps, seconds per step etc.) should be programmed into the Arduino.


\subsection{Bi-directional communication with the computer (Advanced)}
Until now we only read values from the device, but we can also write commands from the computer to the Arduino using the serial interface.
\begin{itemize}
	\item instead of using the potentiometer to set the threshold, implement a way to set the threshold via serial commands
	\item one can use the SCPI protocol as a reference, e.g. set the threshold to 32 degrees with the command "SET:THR 32"
	\item which other commands are necessary to run the measurements from above section without hardcoding the measurement programme (e.g. number of steps, seconds per step etc.)?
	\item write a Python script to perform the measurements from above section via sending SCPI style commands to the Arduino
\end{itemize}



\subsection{Building a PID Controller (Advanced)}
To prevent your resistor from overheating you will now implement a \ac{PID} controller to regulate it's temperature.
\begin{itemize}
	\item build a setup including the heater and the fan
	\item measure the temperature of the resistor with your sensor
	\item write a sketch (\path{TControl.ino}) that uses \ac{PID} to regulate the temperature of the resistor to a stable value
\end{itemize}












