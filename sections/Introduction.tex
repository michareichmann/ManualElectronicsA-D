\section{Introduction}
Arduino is a computer company, project and user community based on easy-to-use hardware and software, that designs and manufactures single-board microcontrollers and microcontroller kits for building digital devices and interactive objects that can sense and control objects in the physical and digital world. All products are distributed as open-source hardware and software, and it's licences permit the manufacture of Arduino boards and software distribution by anyone. The boards are commercially available in preassembled form, or as \ac{DIY} kits.\par

The Arduino project started in 2003 as a program for students at the Interaction Design Institute in Ivrea (Italy), aiming to provide a low-cost and easy way for novices and professionals to create devices that interact with their environment using sensors and actuators. The actual name Arduino comes from a bar in Ivrea, where some of the founders of the project used to meet. The bar was named after Arduin of Ivrea, who was the margrave of the March of Ivrea and King of Italy from 1002 to 1014.\par

In order to work with the Arduino Boards the Arduino programming language, which is based on Wiring, and the Arduino Software (\ac{IDE}), based on Processing are used.

\subsection{Arduino Board}
Arduino boards are able to read inputs - light on a sensor, a finger on a button, or a Twitter
\wrapfig[r]{uno}{5cm}{Arduino Uno board.}
message - and turn it into an output - activating a motor, turning on an LED, publishing something online. You can tell your board what to do by sending a set of instructions to the microcontroller on the board.

\subsection{Grove Shield}

\subsection{Transistor}

